%
%
%
% ██╗    ██╗ ██████╗ ██╗████████╗███████╗██╗  ██╗
% ██║    ██║██╔═══██╗██║╚══██╔══╝██╔════╝██║ ██╔╝
% ██║ █╗ ██║██║   ██║██║   ██║   █████╗  █████╔╝
% ██║███╗██║██║   ██║██║   ██║   ██╔══╝  ██╔═██╗
% ╚███╔███╔╝╚██████╔╝██║   ██║   ███████╗██║  ██╗
%  ╚══╝╚══╝  ╚═════╝ ╚═╝   ╚═╝   ╚══════╝╚═╝  ╚═╝
%
%
%
% .##.......####...######..######..##..##.
% .##......##..##....##....##.......####..
% .##......######....##....####......##...
% .##......##..##....##....##.......####..
% .######..##..##....##....######..##..##.
%
%
%
\documentclass[10pt,american]{scrartcl}
\usepackage{mathpazo}
\usepackage{eulervm}
\renewcommand{\familydefault}{\rmdefault}
\usepackage[T1]{fontenc}
\usepackage[utf8]{inputenc}
\usepackage[a4paper]{geometry}
\geometry{verbose,tmargin=1cm,bmargin=1cm,lmargin=1cm,rmargin=1cm}
\pagestyle{empty}
\setlength{\parskip}{\medskipamount}
\setlength{\parindent}{0pt}
\usepackage{babel}
\usepackage{float}
\usepackage{enumitem}
\usepackage{amsmath}
\usepackage{amssymb}
\PassOptionsToPackage{normalem}{ulem}
\usepackage{ulem}
\usepackage{microtype}
\usepackage[unicode=true,
 bookmarks=true,bookmarksnumbered=true,bookmarksopen=false,
 breaklinks=false,pdfborder={0 0 1},backref=false,colorlinks=false]
 {hyperref}
\hypersetup{
 pdfauthor={Marcio Woitek}}
\makeatletter
\newlength{\lyxlabelwidth}
\usepackage[none]{hyphenat}
\makeatother

\begin{document}

\section*{Gradient Descent Algorithm}

\subsection*{Relevant Formulas}

\uline{Notation}
\begin{itemize}
\item The function we want to minimize is denoted by $f$.
\item The number of independent variables of $f$ is denoted by $n$.
\item The vector whose components are the independent variables of $f$
is written as
\[
\mathbf{x}=\begin{bmatrix}x_{1}\\
x_{2}\\
\vdots\\
x_{n}
\end{bmatrix}.
\]
\item The algorithm produces a sequence of such vectors. This sequence is
denoted by $\mathbf{x}^{\left(0\right)},\mathbf{x}^{\left(1\right)},\mathbf{x}^{\left(2\right)},\ldots$
\end{itemize}
\uline{Iteration Formulas}
\begin{itemize}
\item Algorithm with a \textbf{constant step size}:
\begin{itemize}
\item In this case, we have a single value $\gamma$ for the step size.
\item We can generate the sequence $\mathbf{x}^{\left(0\right)},\mathbf{x}^{\left(1\right)},\mathbf{x}^{\left(2\right)},\ldots$
by using the equation
\[
\mathbf{x}^{\left(i+1\right)}=\mathbf{x}^{\left(i\right)}-\gamma\nabla f\left[\mathbf{x}^{\left(i\right)}\right],\quad i=0,1,2,\ldots,
\]
where $\mathbf{x}^{\left(0\right)}$ must be passed as an input.
\end{itemize}
\item Algorithm with an \textbf{adaptive step size}:
\begin{itemize}
\item In this case, we have a sequence of values for the step size, i.e.,
$\gamma_{0},\gamma_{1},\gamma_{2},\ldots$
\item We can generate the sequence $\mathbf{x}^{\left(0\right)},\mathbf{x}^{\left(1\right)},\mathbf{x}^{\left(2\right)},\ldots$
by using the equation
\[
\mathbf{x}^{\left(i+1\right)}=\mathbf{x}^{\left(i\right)}-\gamma_{i}\nabla f\left[\mathbf{x}^{\left(i\right)}\right],\quad i=0,1,2,\ldots,
\]
where $\mathbf{x}^{\left(0\right)}$ and $\gamma_{0}$ must be passed
as inputs.
\item The step size $\gamma_{i}$ can be written in terms of $\mathbf{x}^{\left(i\right)}$
and $\mathbf{x}^{\left(i-1\right)}$ as
\[
\gamma_{i}=\frac{\left|\left[\mathbf{x}^{\left(i\right)}-\mathbf{x}^{\left(i-1\right)}\right]^{T}\left\{ \nabla f\left[\mathbf{x}^{\left(i\right)}\right]-\nabla f\left[\mathbf{x}^{\left(i-1\right)}\right]\right\} \right|}{\left|\nabla f\left[\mathbf{x}^{\left(i\right)}\right]-\nabla f\left[\mathbf{x}^{\left(i-1\right)}\right]\right|^{2}},\quad i=1,2,\ldots
\]
\end{itemize}
\end{itemize}

\subsection*{Example}

Using Python, I implemented both versions of the gradient descent
algorithm.\\To test my code, I chose to minimize the function
\[
f\left(\mathbf{x}\right)=\frac{1}{2}\sum_{i=1}^{n}\left(x_{i}-i\right)^{2}.
\]
It is straightforward to show that the gradient of this function is
\[
\nabla f\left(\mathbf{x}\right)=\begin{bmatrix}x_{1}-1\\
x_{2}-2\\
\vdots\\
x_{n}-n
\end{bmatrix}.
\]
This equation allows us to conclude that the minimum of $f\left(\mathbf{x}\right)$
is at
\[
\mathbf{x}=\begin{bmatrix}1\\
2\\
\vdots\\
n
\end{bmatrix}.
\]

\end{document}
